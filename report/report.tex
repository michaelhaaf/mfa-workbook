\documentclass[twoside,11pt]{article}

% Any additional packages needed should be included after jmlr2e.
% Note that jmlr2e.sty includes epsfig, amssymb, natbib and graphicx,
% and defines many common macros, such as 'proof' and 'example'.
%
% It also sets the bibliographystyle to plainnat; for more information on
% natbib citation styles, see the natbib documentation, a copy of which
% is archived at http://www.jmlr.org/format/natbib.pdf

\usepackage{jmlr2e}
% Definitions of handy macros can go here
\usepackage{subfig}
\newcommand{\dataset}{{\cal D}}
\newcommand{\fracpartial}[2]{\frac{\partial #1}{\partial  #2}}

% Heading arguments are {volume}{year}{pages}{submitted}{published}{author-full-names}

% \jmlrheading{1}{2000}{1-48}{4/00}{10/00}{Marina Meil\u{a} and Michael I. Jordan}

% Short headings should be running head and authors last names

% \firstpageno{1}

\begin{document}
\title{MSc. Project Report Winter 2022}
\ShortHeadings{MSc. Project Report Winter 2022, McGill University}{}
\author{Author: Michael Haaf \textit{(michael.haaf@mail.mcgill.ca)} \\ Supervisor: Morgan Sonderegger \textit{(morgan.sonderegger@mcgill.ca)}}

\maketitle


\section{Intro (purpose of work, overview main experiments/systems built, results)}

(Content: Introduce speech analysis NLP broadly. explain difference in state of the field between Text NLP (more common, canonical solutions) and speech NLP. This project is a set of experiments, modules, and scripts which demonstrate and overcome difficulties in data management for speech analysis NLP, especially given non-CS background of most practitioners in the field.)


\section{Literature Review}

(briefly summarize state of the field, where else experiments like this can be found with PolyglotDB or similar software. Establish that work builds on MMCauliffe experiments, explain reproduction introuce contributions)

\section{Speech Analysis Pipeline}

Technical explanation of the problems raised in previous sections. In this section I explain the problem overall, showing where canonical solutions exist and indicating where my project makes contributions to explain in subsequent sections (particularly: 3.2/3.3/3.4/3.5/3.6). If we have time, hopefully I can make a quick contribution for 3.7, otherwise I'll point the way for future work.


\subsection{acquiring/storing large corpora}
\subsection{preparing TextGrids}
\subsection{audio signal processing}
\subsection{pronunciation dictionary derivation}
\subsection{MFA alignment}
\subsection{store for experiment using PolyglotDB}
\subsection{linguistic analysis of results with RMD}


\section{Alignment Procedure Implementation}

Implementation details for getting corpuses ready for alignment (scripts, new src modules with flexible business logic for configuration by non experts, unit testing, etc.)

\section{PolyglotDB Package Management}

Implementation details of transition to conda, multiple architectures, integration into development and production usage.

\section{Experiments}

Overview of experiments reproduced/conducted.

\section{Conclusions}

Summarize results and contributions, indicate future work, acknowledgements, etc.

\section{Appendix (TODO: properly)}

"Program listings should only be attached as appendices."

\section{Detailed description of the repository (TODO: add other repos worked on)}
\label{sec:repository}

Experiment code can be viewed here: \\ \href{https://github.com/michaelhaaf/mfa-workbook}. The following describes the usage of each component of this code:

\begin{verbatim}
 |-requirements.txt     # External libraries to be installed
 |-test                 # Unit test files
 | |-
 |-scripts    
 | |-
 |-experiments  # PolyglotDB experiments/reproductions/README
 | |-
 |-sample-data
 | |-
 |-src               # Directory containing contributed modules
 | |-
 |-README.md
\end{verbatim}





% \vskip 0.2in
% \bibliography{sample}

\end{document}